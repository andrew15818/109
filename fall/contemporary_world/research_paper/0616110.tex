\documentclass{article}

% double-spacing
\usepackage{setspace}
\doublespacing

\usepackage[margin=1.0in]{geometry}
\usepackage{xeCJK}
\setCJKmainfont{WenQuanYi Micro Hei Mono}

% indent each paragraph
\usepackage{indentfirst}

% Change this line for ref style
\usepackage[backend=bibtex, style=ieee]{biblatex}
\bibliography{0616110.bib}

%\addbibresource{0616110.bib}

\author{Andr\'es Ponce \\
\and
0616110 \\
\and
彭思安
}
\title{Taiwan and Statehood}
\date{\today}

\begin{document}
\maketitle
International affairs are often governed by politics of the major powers, ongoing wars,
or other fast-paced and increasingly ``scandalous'' stories. They usually have fast-developments
and the story changes often: a city gets captured in an armed conflict; a resolution 
passes by a narrow margin and vocal opposition; or something shocking happens in a 
piece of media. 
However, a particular issue that has occurred for much longer often goes ignored. 
The Republic of China's\footnote{Or is it Taiwan? Even a name can be a source of conflict. I try my best
to use Taiwan for post 1949 and ROC for pre 1949.}
statehood represents, personally, one of the most fascinating global issues. 
In class, the topic of statehood was one of the first ones discussed, and this essay 
presents this simple topic applied to a very close and personal. It also serves as an opportunity 
to investigate and learn about the history of a place I currently call home.

In a more general sense, the topic of Taiwan's statehood is an important one. Taiwan's
influence extends far beyond its own borders, especially in key industries such as semiconductors,
but also in other aspects such as human rights and freedoms. It is also important to have a clear
discussion of whether Taiwan is a sovereign state or not because, if we conlcude that it is not so, 
the land would have to be under control of some other entity, given that millions currently reside 
on the islands. The consensus for Taiwan's statehood would also have to be applied to other similar 
situations in the future, which adds to the importance of a correct assessment in the present.

\section{Historical Background}
The story of the Republic of China begins after the Xinhai Revolution (辛亥革命)~\cite{ChineseRevolution}.
This event led to the overthrow of the Qing dynasty, and to the Republic of China
under the leadership of Sun Yat Sen (孫中山). 
In the 1920s, due to the expulsion and 
assasination of members of the Chinese Communist Party\cite{ChineseCivilWar} (which up to that point had
been a part of the KMT) the Chinese Civil War started in full force. 
When the main priority of the KMT government under Chiang Kai-Shek (蔣介石) turned to 
fighting the Japanese, the conflict with the CCP stopped. After victory over the 
Japanese, the conflict quickly resumed and continued until the KMT's defeat in 1949
and establishment of the Republic of China in Taiwan.

After relocation of the ROC to Taiwan, many major powers still maintained formal
relations with the ROC\@. During and after the 1970s, most countries changed their
official recognition from Taiwan to the People's Republic of China as the only 
representative of ``China''. Since then, while most major countries hold unofficial 
relations with Taiwan, for example having trade offices\footnote{These trade, or cultural offices, serve as functional embassies.}
or even defense promises~\cite{TaiwanRelationsAct}, Taiwan's role in the world stage has become increasingly ambiguous.

\section{Argument and Analysis}

Having given a (very) brief overview of the historical circumstances that led to 
Taiwan's current predicament, let us focus on what it even means to be a state. This 
question is crucial for how we tackle the Taiwan issue.
If the characteristics of sovereign states can be enumerated and it can be shown that Taiwan
posseses these characteristics, the case for Taiwan's statehood will be much stronger.

In 1933, the Montevideo Convention on the Rights and Duties of States, attended by member countries of the 
Organization of American States, put into use the declarative theory of statehood~\cite{MontevideoConvention}. 
This view of nation states,
according to article 1 of the convention, states should posses four characteristics: a defined territory, a permanent population,
a government, and the capacity to enter into relations with other states~\cite{MontevideoConventionText}.

While these requirements are not set in stone, they have been the basis of international law for many decades.
Taiwan fulfills these requirements. On point the first point above, it is clear Taiwan posseses a defined territory. 
Besides the main island of Taiwan, other outlying islands such as Green Island, Penghu, and Kinmen, among others. 
Whatever territory the constitution or any other document claims belong to the Republic of China in the mainland,
that does not change that the territory now under the control of the government exists
and is clearly defined.

For the second point, the population is also permanent. Although people are constantly 
moving in and out of Taiwan, the vast majority of the population lives permanently in the
territory oultined above. This means that Taiwan is suited to hold a permanent population
based on its resources. 

Taiwan also possesses a functioning government. The rule of law exists according to the 
principles laid out in the constitution and other documents. Various government agencies
are responsible for managing different state functions: the Legislative Yuan, the Ministry
of Foreign Affairs, etc\dots

Lastly, Taiwan can enter into relations with other states. Most evidently are the countries
that continue to have formal relations with Taiwan. A mere territory or a land controlled
by another state could not enter into such agreements. Taiwan has embassies, cultural offices,
defense and cooperation agreements, and many other such relations. Some level of autonomy
is required to decide who to do business with, who to sign agreements with, and who to 
buy weapons from.

These four characteristics  are easily possessed by Taiwan. There still remains another 
aspect that many cite as a reason why Taiwan cannot be an independent state: international
recognition. The declarative theory explicitly mentions that statehood is independent of 
recognition by other states, however in the era of instutions such as the United Nations,
this is still an area of concern for many. Many cite participation in global organizaitons
as a true sign of statehood, however this standard is too loose for many other states that
are universally regarded as sovereign states. For example, the Republic of China was regarded by many countries
as the legitimate representative of China for some decades after 1949. However, during 
this time the People's Republic of China exerted control over mainland China, albeit with
little formal diplomatic relations.

On the other hand, the PRC government often justifies its control of Taiwan by citing 
UN Resolution 2758, which claims that the PRC is the sole representative of China, and 
this includes Taiwan~\cite{krasner2001problematic}. The claim goes that if the Communist Party was the victor of the 
Chinese Civil War, then all lands formerly belonging to the Republic of China~\footnote{This 
would include Taiwan since it was returned to the ROC after the second Sino-Japanese
War.} belong to the PRC\@. However, this assumes that the Republic of China does not exist
anymore, which is exactly what is being discussed. If the ROC government had never 
relocated to Taipei and had instead disbanded in Nanjing, this claim would likely 
hold. However, given the current situation, this position begs the question of the 
Republic of China's existence.

Despite the lack of international recognition, 
very few people would claim that the People's Republic was not a legitimate country
during this time before its acceptance to international organizations in the 1970s. This 
standard cuts both ways, and by saying that Taiwan is not a sovereign state because it is 
not a part of such organizations, the same could be said of all countries before their 
governemnts decided to join international organizations. For example, Poland was not present
at the founding of the United Nations~\cite{UNHistory} but joined the UN shortly thereafter, so was Poland not a country before 
it joined? The implications of this question would redefine our entire notion of a sovereign state,
which is exactly what we are trying to prove.

If Taiwan fulfills the unofficial condiitons layed out in the Montevideo Convention and 
the dependence on international recognition presents some problems, is there another way to show
that Taiwan is not a sovereign state? In my mind, there would have to be some fundamental characteristic
of statehood that every single state has yet Taiwan lacks. If this requirement is not one 
of the previously mentioned, it remains to be found.

\section{Conclusion}
In conclusion, there is a large amount of evidence for the claim that Taiwan is a sovereign and 
independent state. Decades of international support, combined with the autonomy of any other 
nation, and continued (yet limited) international support seem to affirm the statehood of Taiwan.
A troubled history and an influential opponent cannot change this fact.

This issue, although sensitive in some circles, remains important for several reasons. Firstly,
Taiwan holds an influential position in key industries, such as semiconductors. Next, whatever
our stance on Taiwan's statehood has to determine the standard for statehood going forward, in case
future people also find themselves in a similar situation. Although not the most thrilling event, 
and certainly an issue so subtle as to require a much longer examination, it nevertheless remains 
one of the most fascinating topics currently unfolding.

\printbibliography
\end{document}
