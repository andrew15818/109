\documentclass{tufte-handout}
\usepackage{amsmath}
\usepackage{graphicx}

\title{Selected Topics in Visual Recognition and Deep Learning}
\author{Andr\'es Ponce}

\begin{document}
\maketitle

\begin{abstract}
	Computer vision deals with connecting a camera to a computer and
	trying to process an image data. The idea is also to simulate 
	human vision using some algorithms and deep learning.
\end{abstract}

An \textbf{image} is a collection of $(x,y)$ points which contain some 
intensity value.
In a black and white image, the intensity vlaue might be just how dark or
light it is at the point $(x,y)$, where we start counting from the top left.

If we want to detect some pattern in the image, we would have to perform a 
\textbf{convolution}.
\footnote{A \textbf{convolution} is a function that describes how two 
\textit{functions} influence each other.} 
So here, one we might have a linear combination of the surrounding pixels.
Then we use a \textbf{filter} to somehow transform the images.
An example filter might be 

\begin{center}
	\[
	\begin{bmatrix}
		1 & 0 & -1\\
		1 & 0 & -1\\
		1 & 0 & -1
	\end{bmatrix}
	\]
\end{center}

If we pass this filter through an image, what will the result be? 
If we just take the total sum, we are going to take the difference between the 
pixels on the left and right side.
This difference might be useful in detecting edges and such.

%\[\left[ \begin{array}{cc}
%		3 & 0 & 1 & 2 & 7 & 4 \\
%		1 & 5 & 8 & 9 & 3 & 1 \\
%		2 & 7 & 2 & 5 & 1 & 3 \\
%		0 & 1 & 3 & 1 & 7 & 8 \\
%		4 & 2 & 1 & 6 & 2 & 8 \\
%		2 & 4 & 5 & 2 & 3 & 9 
%
%	\end{array} \right]
%\left [ \begin{array}{cc}
%		1 & 0 & -1 \\
%		1 & 0 & -1 \\
%		1 & 0 & -1 
%	\end{array} \right]
%	= 
%	\left \begin{array}{cc}
%			-5 & -4 & 0 & 8 \\
%			-10 & -2 & 2 & 3 \\
%			0  & -2 & -4 & -7 \\
%			-3 & -2 & -3 & -16
%		\end{array} \right
%\]

If we mess around with the filter structure, we can either apply a \textbf{blur} 
effect or a sharpening effect.

Some of the difficulties in deep learning occur from some artifacts which 
obfuscate the image. 
For example, \textbf{illuminations}, deformations, and occlusions.

\subsection{Conventional Approach to Object Recognition}
In the traditional approach to object recognition, there are several steps:
\begin{enumerate}
	\item Image Collection
	\item Feature Extraction
	\item Classifier Training
	\item Trained Classifier
\end{enumerate}

The main image characteristic is \textbf{features}, which are 
just characteristics of the images that are of interest to us.
Maybe the first layers learn some low-level, courser feautres, and 
higher layers learn finer features of the image.

How many layers and what type do we add in each layer?
While it may not be an exact science, sometimes the error \textit{increases}
as the number of layers increases.

Some other functions for visual problems involving deep learning include:
\begin{enumerate}
	\item \textbf{Image Alignment}: Try to find the common image in two images.
	\item \textbf{Image Matching}: Find common regions in different images.
	\item \textbf{Image Stitching}: Find a panorama picture by joinging multiple images :D 
\end{enumerate}

\end{document}
