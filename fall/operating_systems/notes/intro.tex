\documentclass{tufte-handout}
\usepackage{graphicx}
\graphicspath{ {./img/} }
\title{Introduction to Operating Systems}
\author{Andr\'es Ponce}

\begin{document}
\maketitle
\begin{abstract}

\section{Introduction}
An \textbf{operating system}(OS) is a piece of software that lies between the
hardware and software layer inside a computer. 
The OS helps the machine allocate resources to processes that require them.
\end{abstract}

The OS allocates resources to different processes on a computer. 
\subsection{Interrupts}
When we have an I/O request, the device controller will receive a signal to load 
and move the data to the CPU or main memory. When this process is done, the controller
will send a signal to the CPU informing of a successful completion of the transfer.

There are wother ways of triggering a system interrupt. Sometimes the program might
create an interrupt, known as a \textbf{software interrupt}. Other times, the 
hardware itself might create an interrupt. When this happens, there is a wire
connected to the CPU which is directly activated.

\subsection{Computer System Architecture}
In days past, the comptuer was able to run on a single \textbf{core}, which 
executed a general instruciton set needed by application programs. 
There could also be some smaller and different processors such as 
\textbf{controllers}.
\footnote{A \textbf{controller} is a type of device which handles taking information
from I/O devices and getting it to the CPU. A controller usually has its own buffer,
where it stores the data it is responsible for fetching before sending it to memory or 
CPU.}

However, recently there are multicore systems, which have more than one physical core.
This means that they can execute more than one instruction at at a time. 
However, it depends on whether there are any resources to allocate available.

There are a couple useful distinctions:
\begin{itemize}
	\item Random Access: This type of memory allows any part of its data to be accessed
			without having to first access the other parts. This memory is 
			\textit{volatile}, which means it will be wiped clean next time the computers
			boot.
	\item Solid State: A solid state drive(SSD) uses solid state technology and removes
			any moving components from the drive. This allows for faster read speeds since
			we don't have to wait for the platter to spin in the correct decision 
			under the read/write head.
\end{itemize}

\begin{center}
	\begin{figure}
		\includegraphics[scale=0.3]{mem_hierarchy}
		\caption{Memory hierarchy. As the speed of the memory increases, so does its
			cost. So we can only include a limited amount of very quick memory. Cache
			and registers usually are very limited and are located directly in the CPU.}
	\end{figure}
\end{center}

\end{document}
