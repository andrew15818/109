\hypertarget{virtual-memory}{%
\section{Virtual Memory}\label{virtual-memory}}

This refers to when the program's memory is not all necessarily in main
memory.

\hypertarget{demand-paging}{%
\subsection{Demand Paging}\label{demand-paging}}

This refers to only bringing the pages that we need into the memory.
Since some code will not be executed often, it might not be the most
efficient to load the entire program if we don't need all of it.

The main process for entering a new program into the page table occurs
when we have a \emph{page fault}, or when we want to access a page and
it is currently not in the process's address space.

The logical extreme of doing demand paging is that we can start
executing a program with no pages in memory, and immediately fault on
the first page call. Demand paging also requires the page table, so we
mmark whether a page is accessible by a certain process. The
\emph{valid-invalid bit} is accessible through every process's frame,
and keeps track of which frames are acessbile by the operating system.

Our operating system also keeps a list of the free frames that are
available for us to use. When a process needs to expand in memory by
requesting another frame, we take one from that list and assign it to
the process, expanding the page table in the process.

How long does it take to access the memory? It depends on the frequency
of the memory page faults. If \emph{p} is the probability of frame
drops, then we would have to know the time it takes us to normally
access the memory. We could then say the \textbf{effective access time}
is \emph{(1 - p)ma x page fault time}.

The process for executing a page fault is a long one, since we have to
notify the operating system, we have to check that the page reference
was legal (valid-invalid bit), context switch while the page is being
returned. Then we have to context switch back again into our
application.

\hypertarget{copy-on-write}{%
\subsection{Copy-on-Write}\label{copy-on-write}}

We had just seen how, using demand paging, we could start execution of a
process just by knowing the first page and defaulting on it. However,
this might be uneccessary. When creating a child process, we would copy
all the parent's address space for the child. However, we can just share
the pages between the parent and the child process. If we set these
pages as \emph{copy on write}, then every time either process wants to
change a page, we can duplicate it for the other process.

Likewise, the \texttt{vfork()} system call will create a new child
process but it will be able to edit the parent's pages. These changes
will then be visible by the parent. Since no copying of pages takes,
place, this method can be quite efficient.

\hypertarget{page-replacement}{%
\subsection{Page Replacement}\label{page-replacement}}

The idea behind page replacement is how we can get rid of pages that are
currently in memory. For example, if there are no free frames in the
system memory, then we can take out the one that has been used least
recently, or the one that was brought in a long time ago. We would have
to updat ethe page table (and other tables) to indicate that the page
has been removed from memory.

The overall process for replacing a frame is as follows: 1. We find the
location of the second page in secondary storage. 2. Find a free frame
in memory. If there are none, we select a \textbf{victim frame} with the
help of a page-replacement algorithm. Otherwise if we find a free frame,
we directly use it. 3. Change the page tables to indicate that we
replaced a frame.

Notice that if there were no free frames, we would need to page-out the
victim frame and page-in the replacement frame, which would effectively
double the memory access time. To this end, we place a \textbf{dirty
bit} in every page. This bit is toggled whenever the page is written
into while in memory. When we replace the frame, we check if the frame
has been updated since it was brought in, and if it has been we have to
write it back into secondary storage.

Page replacement is essential to demand paging. The virtual addresses
used by the CPU no longer need to be mapped directly to physical
addresses. The set of virtual addresses can be much longer than physical
memory if we can replace a page on demand. When a process wants to
access a frame, we can bring it in from secondary storage if the page is
not in memory. However, the pages that the program requires need to be
in memory.

\hypertarget{page-replacement-algorithms}{%
\subsubsection{Page-Replacement
Algorithms}\label{page-replacement-algorithms}}

Since bringing in pages from I/O is expensive, requiring possibly
hundreds of cycles, we should design and choose our page-replacement
scheme so as to minimize page faults.

\hypertarget{fifo-page-replacement}{%
\subsubsection{FIFO Page Replacement}\label{fifo-page-replacement}}

We can also keep a queue with the pages that have been brought into
memory. This means that when we have to load a new page into memory, we
replace the one next in line at the queue. This means that we replace
the page that was brought in the longest time ago.

\textbf{Example} 

\($\alpha$)
