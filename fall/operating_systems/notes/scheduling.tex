\documentclass{tufte-handout}
\title{Chapter 5: CPU scheduling}
\author{Andr\'es Ponce}
\usepackage{graphicx}
\graphicspath{ {./img} }

\begin{document}
\maketitle
\begin{abstract}
CPU scheduling refers to the process by which we decide which
programs to run. One of the most important OS functions is scheduling
programs, so allocating one of the most important system resources
is a challenging task. 
\end{abstract}

\section{Idea}
The CPU works in what are called \textbf{bursts}. Bursts are a short period of time
where the CPU is executing a certain type of function. The CPU usually works with 
\textbf{CPU bursts}, followed by \textbf{I/O bursts}. Usually program execution alternates
between the two.

When does scheduling take place? Depends on the OS. There are usually four conditions where
that happens:
\begin{itemize}
	\item When the state of the program changes from running state to waiting.
	\footnote{Maybe b/c of I/O or a child process running.}
	\item When the process goes from running state to ready state(e.g. when interrupt occurs).
	\item When the process goes from waiting state to ready state(e.g. I/O completion.)
	\item Process termination.
\end{itemize}

When the first or the fourth condition occurs, we say the scheduling is \textbf{non-preemptive},
since essentially the OS waits for the program to either terminate or to be idle to perform 
a context switch. The other conditions cause what is called \textbf{preemptive} scheduling. 
This means the program can be in the middle of execution when it is rescheduled and another
program is loaded.

This can result in \textbf{race conditions}, for example if the data is required by two
programs, and gets changed during execution.


\end{document}
