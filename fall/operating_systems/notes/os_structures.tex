\documentclass{tufte-handout}
\usepackage{graphicx}
\graphicspath{ {./img} }

\title{Chapter 2: Operating System Structures}
\author{Andr\'es Ponce}

\begin{document}
\maketitle
\begin{abstract}
	An \textit{operating system} allows us to allocate resources
	to a machine. We can use either a graphical environment or 
	all from the \textit{command line}.
\end{abstract}


\section{Operating System Services}
The OS has some key services that it provides:

\begin{itemize}
	\item \textbf{User Interface}: How does the user interact with the system? There 
			have traditionally been two ways, \textbf{command-line interface}, where the 
			user types the commands it wants the computer to execute. There is also the 
			option for a \textbf{user interface}, where the user clicks icons and opens
			graphical programs to run commands and operate the computer.

	\item \textbf{Program Execution}: One of the main functions of an operating system 
			is to load programs into memory and run those programs. One of the main 
			abstractions that the OS provides is to load/execute programs.

	\item \textbf{I/O operations}: For safety reasons, the user seldom interacts directly
			with I/O devices, but the computer has to communicate with the outside. 
			Writing to a network interface or talking with the filesystem maybe should not
			be left to the user.....
			
\end{itemize}

\end{document}
